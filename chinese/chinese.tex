% Copyright 2019 Clara Eleonore Pavillet

% Author: Clara Eleonore Pavillet (original author), Leonard Quentin Marcq (modified for Tsinghua), 赵宗义 (参与了修改)
% Description: This is an unofficial Tsinghua University Beamer Template I made from scratch. Feel free to use it, modify it, share it.
% Version: 1.1


\documentclass{beamer}
\usepackage{ctex}
\usepackage{fontspec}

\usetheme{Madrid}
\usecolortheme{default}

%%%<<<---
\newcommand{\system}{HashFlow}
\newcommand{\variable}{\mathrm}
\newcommand{\field}{\mathsf}
%%%--->>>


\title{探索实践}
%\titlegraphic{\includegraphics[width=2cm]{Theme/Logos/tsinghua_emblem_dark.png}}
%\author{王德胜}
\author{张三}
%\institute{清华大学计算机科学与技术系}
\date{} %\today

\logo{\includegraphics[height=1cm]{logo}} 

\begin{document}

{\setbeamertemplate{footline}{} 
\frame{\titlepage}}



\begin{frame}{{\system}的总体架构}
\textbf{控制平面}
\begin{enumerate}
\item 流记录项的分类和聚合
\item 流记录项的存储和分析
\end{enumerate}
\textbf{数据平面}
\begin{enumerate}
\item 哈希冲突解决策略
\item 大流检测机制
\item 不活跃流退出机制
\end{enumerate}
\end{frame}

\begin{frame}{数据平面}
\begin{enumerate}
\item 数据平面包含了两个哈希表, 分别是主表 (Main Table) 和辅表 (Ancillary Table).
\item 主表中每个哈希桶包含两个域: $\field{fieldID}$和$\field{count}$
\item 辅表中每个哈希桶也包含两个域: $\field{digest}$和$\field{count}$
\end{enumerate}
\begin{figure}
	\centering
	\includegraphics[width=0.6\linewidth]{figures/datastructure}
\end{figure}

\end{frame}

  
\begin{frame}{}
\begin{center}
    \textbf{谢谢!}
\end{center}
\end{frame}


\end{document}

